\documentclass[main.tex]{subfiles}

\begin{document}
\section{Response-guided Principal Component Classification (RgPCC)}

First, a note on notation. We let $p$ denote the number of predictors in our data, $p_{\lambda}$ to a non-decreasing penalty function and $p_i$ to be the $i^{\text{th}}$ component of a vector of probabilities \gls{prob}.

Suppose we have a binary classification problem with design $\X$ and classes $\y$. In logistic regression we fit by maximizing the log-likelihood, or equivalently, by minimizing the negative log-likelihood. We propose that we penalize this objective function to get the following problem:

\begin{align}
	\bbeta^{\text{RgPCC}} = \argmin_{\bbeta} - \ell(\X, \bbeta) + \sum_{j = 1}^p p_{\lambda}(\beta_j) \label{RgPCC_obj}
\end{align}

\subsection{Method}
To solve (\ref{RgPCC_obj}), we can approximate the log-likelihood $\ell(\X, \bbeta)$ with a quadratic approximation centered at $\bbeta^*$ (the minimizer of $- \ell(\X, \bbeta)$). This quadratic approximation can be interpreted as the least squares error

\begin{align}
	- \ell(\X, \bbeta) \approx ||\W^{1/2} (\z - \X \bbeta)||^2 = || \widetilde{\y} - \widetilde{\X} \bbeta ||^2
\end{align}

where $\W$, and $\z$ are as in (\ref{pdef} - \ref{zdef}) with (\ref{pdef}) evaluated at $\bbeta = \bbeta^*$ and $\W^{1/2} \X = \tilde{\X}$, $\tilde{\y} = \W^{1/2} \z$. The derivation of this approximation can be found in \cite{wangleng}. Note that, we may often refer to $\tilde{\X}$ and $\tilde{\y}$ as the \emph{pseudo data} and \emph{pseudo response} respectively.

Equation (\ref{RgPCC_obj}) can be rewritten as an approximate RgPCR problem

\begin{align}
	\argmin_{\bbeta} || \widetilde{\y} - \gls{pseudoX} \bbeta ||^2 + \sum_{j = 1}^p p_{\lambda}(\beta_j) \label{RgPCC_obj2}
\end{align}

For particular choices of $p_{\lambda}$, this may be solved using techniques from \cite{langzou}. That is, by viewing the above in terms of principal components we may solve the equivalent problem

\begin{align}
	\argmin_{\bgamma} ||\tilde{\y} - \tilde{\U} \bgamma ||^2 + \sum_{j = 1}^p p_{\lambda} \left (\frac{\gamma_j}{d_j} \right )
\end{align}

where $\tilde{\X} = \tilde{\U} \tilde{\D} \tilde{\V}^T$ is the SVD decomposition of $\tilde{\X}$ and $\bgamma = \tilde{\D} \tilde{\V}^T \bbeta$. The matrices $\tilde{\U}$ and $\tilde{\V}$ are orthogonal matrices and are $N \times p$ and $p \times p$ matrices respectively. Let $d_j$ denotes the diagonal entries of $\tilde{\D}$ for $1 \leq j \leq p$. For our description here, we assume that $\tilde{\X}$ has full rank, but the following still holds with a thin SVD decomposition.

As with RgPCR, we may take advantage of the orthogonality of $\tilde{\U}$ and optimize componentwise. That is, with $\hat{\gamma}^{\text{ols}}_j = \y^T \tilde{U}_j$,

\begin{align}
	 \argmin_{\gamma_j} (\hat{\gamma}_j^{\text{ols}} - \gamma_j)^2 + p_{\lambda} \left (\frac{\gamma_j}{d_j} \right ). \label{RgPCC_princ}
\end{align}

From \cite{langzou}, when $p_\lambda(t) = |t|$ is the LASSO penalty, we have the closed form solution of (\ref{RgPCC_princ})

\begin{align}
	\hat{\gamma_j} = \left ( |\hat{\gamma}^{\text{ols}}_j| - \frac{\lambda}{d_j} \right )^+ \cdot \sgn{\hat{\gamma}^{\text{ols}}_j},
\end{align}
We can then rewrite this in the original coordinates using

\begin{align}
	\hat{\bbeta} = \sum_{j = 1}^p \tilde{V}_j \frac{\hat{\gamma}_j}{d_j}.
\end{align}

\subsection{Iterative Approximations}

In practice, a single quadratic approximation of $- \ell(\X, \bbeta)$ may not produce a minimum close to the true minimum. Just as in logistic regression, we take the quadratic approximations iteratively to improve our approximation of the minimum. That is, for some approximate minimizer $\bbeta^{(n)}$, we solve

\begin{align}
	\argmin_{\bbeta} - \ell(\X, \bbeta) + \sum_{j = 1}^p p_{\lambda}(\beta_j)
\end{align}

which can be approximated by

\begin{align}
	- \ell(\X, \bbeta) \approx || \widetilde{\y}^{(n)} - \widetilde{\X}^{(n)} \bbeta ||^2 \label{RgPCC_iter}
\end{align}

where
\begin{align}
	\p^{(n)} &= \dfrac{\text{exp}(\X \bbeta^{(n-1)})}{1 + \text{exp}(\X \bbeta^{(n-1)})} \label{pdef}\\
    \W^{(n)} &= \text{diag}[p_i^{(n)}(1-p_i^{(n)})] \label{Wdef}\\
    \z^{(n)} &= \X \bbeta^{(n-1)} + (\W^{(n)})^{-1}(\y - \p^{(n)}). \label{zdef} \\
	\tilde{\X}^{(n)} &= (\W^{(n)})^{1/2} \X \\
	\tilde{\y}^{(n)} &= (\W^{(n)})^{1/2} \z^{(n)}
\end{align}

As with the first iteration, we can solve

\begin{align}
	\argmin_{\bbeta} || \widetilde{\y}^{(n)} - \widetilde{\X}^{(n)} \bbeta ||^2 + \sum_{j = 1}^p p_{\lambda}(\beta_j)
\end{align}

to estimate the solution to (\ref{RgPCC_iter}). Then, as with the first iteration, we can express the above using principal components

\begin{align}
	\argmin_{\bgamma} ||\tilde{\y}^{(n)} - \tilde{\U}^{(n)} \bgamma ||^2 + \sum_{j = 1}^p p_{\lambda} \left (\frac{\gamma_j}{d_j^{(n)}} \right )
\end{align}

and take advantage of the orthogonality of $\tilde{\U}$ so solve componentwise

\begin{align}
	\argmin_{\gamma_j} (\hat{\gamma}_j^{\text{ols}, (n)} - \gamma_j)^2 + p_{\lambda} \left (\frac{\gamma_j}{d_j^{(n)}} \right ). \label{RgPCC_princ_iter}
\end{align}

\subsection{Algorithm for LASSO penalty}

Here, we will briefly describe the algorithm used to approximate a solution to (\ref{RgPCC_obj}) when we use the LASSO penalty. This will generalize to any penalty that has an RgPCR closed form solution.

\begin{enumerate}[(Step 1)]
	\item Guess an initial $\hat{\beta}^{(0)}$
	\item With $\hat{\beta}^{(0)}$ calculate the following
		\begin{align*}
			\p^{(1)} &= \dfrac{\text{exp}(\X \bbeta^{(0)})}{1 + \text{exp}(\X \bbeta^{(0)})} = \dfrac{\text{exp}(\U \bgamma^{(0)})}{1 + \text{exp}(\U \bgamma^{(0)})}\\
			\W^{(1)} &= \text{diag}[p_i^{(1)}(1-p_i^{(1)})] \\
			\z^{(1)} &= \X \bbeta^{(0)} + (\W^{(1)})^{-1}(\y - \p^{(1)})\\
			\tilde{\X}^{(1)} &= (\W^{(1)})^{1/2} \X \\
			\tilde{\y}^{(1)} &= (\W^{(1)})^{1/2} \z^{(1)}
		\end{align*}
	\item Compute the solution to the RgPCR problem
	\begin{align}
		\argmin_{\bgamma} ||\tilde{\y}^{(0)} - \tilde{\U}^{(0)} \bgamma ||^2 + \lambda \sum_{j = 1}^p  \left | \frac{\gamma_j}{d_j^{(0)}} \right |
	\end{align}
	using the closed form solution
	\begin{align}
		\hat{\gamma_j}^{(1)} = \left ( |\hat{\gamma}^{\text{ols}}_j| - \frac{\lambda}{d_j} \right )^+ \cdot \sgn{\hat{\gamma}^{\text{ols}}_j},
	\end{align}
	where $\hat{\gamma}^{\text{ols}}_j = \tilde{y}^{(1)} \tilde{U}^{(1)}_j$
	\item repeat steps 2 and 3 using $\hat{\bgamma}^{(1)}$ and similarly using $\hat{\bgamma}^{(k)}$ until $\varepsilon^{(k)} = \frac{|| \bbeta^{(k)} - \bbeta^{(k+1)}||}{||\bbeta^{(k)}||} < \varepsilon$ for some desired tolerance $\varepsilon$.
	\item Recover the solution in the original coordinates with $$\hat{\bbeta}^{(n)} = \sum_{j = 1}^p \tilde{V}^{(n)}_j \frac{\hat{\gamma}_j^{(n)}}{d_j^{(n)}}.$$
\end{enumerate}

% \begin{algorithm}[H]
% \SetAlgoLined
% \KwResult{RgPCC}
%  initialization\;
%  \While{While condition}{
%   instructions\;
%   \eIf{condition}{
%   instructions1\;
%   instructions2\;
%   }{
%   instructions3\;
%   }
%  }
%  \caption{How to write algorithms}
% \end{algorithm}

\end{document}